\section{Network Synthesis Software} \label{software}

The software may be compiled in a unix type operating system by going into the SRC/ directory and using the command

make

The makefile may need to be altered for different compilers or to apply specific compilation flags for example. The Makefile in SRC/ is designed to use the f95 gfortran compiler. The executable file, network\_synthesis, is moved to the bin directory.

\subsection{Running network\_synthesis}

The network synthesis process required a command line argument which is the filename of the input impedance function. It is run woith the command

network\_synthesis filename

The input impedance function file may take one of two forms; a pole-residue format or a rational function format as described below.

\subsubsection{Pole-residue format}

The pole-residue input file format contains the pole-residue representation of the input function for the specified model order.
The format of the function uses a normalised angular frequency where the angular frequency normalisation (wnorm) is often based on the
maximum frequency in the input data used in a curve fitting process (for example the VECTOR\_FIT software (https://github.com/chrissmartt/vector\_fit). The function is therefore of the form

\begin{equation}\label{eq:pr1}
Z \left( s \right) = d+\frac{s}{wnorm} h+\sum_{n=1}^N \frac{c_n}{\frac{s}{wnorm}-a_n} 
\end{equation} 

At the end of the file is the (normalised) frequency range and number of samples in the input data.

Below is an example input function in pole-residue form:

\begin{small}
\begin{verbatim}
Vfit filter output
           1  # order
   628318530.71795857       # wnorm
   8.5000000899999470E-002  # d
   27.017695590661077       # h
                   pole (a)                              residue (c)
-1.12842870712061E-021 0.00000000000000   1.59154949383951E-003 0.00000000000000     
        wnorm_min             wnorm_max                   nw
1.0000000000000001E-005   1.0000000000000000              200
\end{verbatim}
\end{small}

\subsubsection{Rational function format}

The rational function impedance takes the form:

\begin{equation}\label{eq:rat1}
Z \left( s \right) = \frac{a_0+a_1 \frac{s}{w_{norm}} + +a_2 \frac{s}{w_{norm}}^2 + \dots}
{b_0+b_1 \frac{s}{w_{norm}} + +b_2 \frac{s}{w_{norm}}^2 + \dots}
\end{equation} 

The file contains the frequency range of validity of the model (often the frequency range for which data is available for a curve fitting process applied to dervive the rational function model e.g. GGI\_TLM\_filter\_fit 
from the GGI\_TLM project (https://github.com/ggiemr/GGI\_TLM)). Following this is the angular frequency normalisation constant then in turn the numerator order, numerator coefficients, denominator order and denominator coefficients.

Below is an example input function in rational function form:

\begin{verbatim}
# rational function impedance representation
0.01 1.0 # fmin, fmax (not normalised)
# Impedance filter
1d0 # wnorm
2   # a order, a coefficients follow below:
2d0 2d0 1d0
2   # b order, b coefficients follow below:
1d0 1d0 1d0
\end{verbatim}



\clearpage
